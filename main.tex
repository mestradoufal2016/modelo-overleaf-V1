%########## tipo de documento ##########
%Sintaxe = documentclass[configuração do papel][tipo de documento]

%Configurações disponiveis: 12pt, 14,pt, a4paper, letterpaper, legalpaper

%Tipo de documento: article, minimal, report, book, slides, letter, beamer, exam

\documentclass[12pt, a4paper]{article}

%#### CONFIGURAÇÃO DE IDIOMA E ACENTUAÇÃO ###

\usepackage[utf8x]{inputenc}
\usepackage[portuguese]{babel}
\usepackage[T1]{fontenc}
\usepackage{indentfirst}

%#### CONFIGURAÇÃO DE GRÁFICOS ###

\usepackage{graphicx}

%#### CONFIGURAÇÕES MATEMÁTICAS ####

\usepackage{amsmath} 
\usepackage{amsfonts} 
\usepackage{amssymb} 

%#### CONFIGURAÇÕES REFERENCIAS BIBLIOGRÁFICAS ####

\usepackage{natbib}

%#### CONFIGURAÇÕES EXTRAS GERAIS ####

\usepackage{multicol} %Usar colunas
\usepackage{color} %%Usar cores
\usepackage{lipsum}
\usepackage{hyperref}
\usepackage{url}
\usepackage{csquotes}


%#### CONFIGURAÇÕES DO AUTOR ####

\title{MODELO GERAL OVERLEAF} %Nome do seu projeto
\author{César Felipe Gonçalves da Silva}
\date{19 de dezembro de 2016}

%### INICIO DO DOCUMENTO ###

\begin{document}

%### PAGINA DE TITULO ###
%### COMENTE SE NAO QUISER PAGINA DE TITULO ###

%\maketitle

\begin{titlepage}
\begin{center}
	
\begin{figure}[!ht]
\centering
\includegraphics[width=0.8\textwidth]{logo-ufal.png} 
\end{figure}

\large{Mestrado em Modelagem computacional do conhecimento}\\ 

\large{Leitura e Escrita de Trabalho Científico}\\ 

\vspace{110pt}
        
\textbf{\LARGE{Gerência e configuração de thin switches}}\\

\vspace{3,5cm}
\end{center}
	
\begin{flushleft}
Aluno: César Felipe Gonçalves da Silva\\
Professor orientador: Professor Doutor Rafael Amorim Silva \\
\end{flushleft}

	
\begin{center}
\vspace{\fill}
Dezembro\\
2016

\end{center}
\end{titlepage}

%### FIM DA SEÇÃO DA PAGINA DE TITULO ###

%### FOLHA DE ROSTO ###
%### COMENTE SE NAO QUISER FOLHA DE ROSTO ###

\begin{titlepage}
\begin{center}
	
\begin{figure}[!ht]
\centering
\includegraphics[width=0.8\textwidth]{logo-ufal.png} \end{figure}

\large{Mestrado em Modelagem computacional do conhecimento}\\ 
\large{Leitura e Escrita de Trabalho Científico}\\ 
\vspace{100pt}

        
\textbf{\LARGE{Relatório}}
\title{\large{Título}}

			
\end{center}
\vspace{1,5cm}
	
\begin{flushright}

\begin{list}{}{
\setlength{\leftmargin}{4.5cm}
\setlength{\rightmargin}{0cm}
\setlength{\labelwidth}{0pt}
\setlength{\labelsep}{\leftmargin}}

\item Primeiro Relatório da Disciplina Leitura e Escrita de Trabalho Científico.

\begin{list}{}{
\setlength{\leftmargin}{0cm}
\setlength{\rightmargin}{0cm}
\setlength{\labelwidth}{0pt}
\setlength{\labelsep}{\leftmargin}}

\item Aluno: César Felipe Gonçalves da Silva\
\item Professor orientador: Professor Doutor Rafael Amorim Silva\
      		

\end{list}
\end{list}
\end{flushright}
\vspace{1cm}


\begin{center}
\vspace{\fill}
Dezembro\\
2016
\end{center}
\end{titlepage}

%### FIM DA SEÇÃO DE FOLHA DE ROSTO ###

%### INICIO DA SEÇÃO DE SUMÁRIO ###
%### COMENTE CASO NAO QUEIRA SUMÁRIO ###

\newpage
\tableofcontents
\thispagestyle{empty}
\newpage

%### FIM DA SEÇÃO DE SUMÁRIO ###

\section{Como fazer abstract}

Abaixo um exemplo de abstract:

\begin{abstract}
texto do abstract - texto do abstract - texto do abstract - texto do abstract - texto do abstract - texto do abstract - texto do abstract - texto do abstract - texto do abstract - texto do abstract - texto do abstract - texto do abstract
\end{abstract}

\newpage
\section{como fazer listas}

\begin{itemize} % Lista pontilhada
\item escreva aqui o primeiro item \\
\item escreva aqui o segundo item \\
\item escreva aqui o terceiro item \\
\end{itemize}

\begin{enumerate}
\item escreva aqui o primeiro item \\
\item escreva aqui o segundo item \\
\item escreva aqui o terceiro item \\
\end{enumerate}

\begin{description}
\item[ITEM 1] escreva aqui o primeiro item \\
\item[ITEM 2]  escreva aqui o segundo item \\
\item[ITEM 3]  escreva aqui o terceiro item \\
\end{description}

\newpage
\section{Como fazer seções}

Acima o comando para iniciar a seção. Neste ponto daqui, é só começar a digitar seu texto

\subsection{Como fazer subseções}

Acima o comando para iniciar a subseção. Neste ponto daqui, é só começar a digitar seu texto

\subsubsection{Como fazer sub-subseções}

Acima o comando para iniciar a sub-subseção. Neste ponto daqui, é só começar a digitar seu texto

\newpage
\section{Como fazer negrito, italico, sublinhado e ênfase}

Aqui um exemplo de \textbf{negrito} \newline
Aqui um exemplo de \textit{Italico} \newline
Aqui um exemplo de \underline{Sublinhado} \newline
Aqui um exemplo de \textbf{\textit{Negrito com italico}} \newline
Aqui um exemplo de \textbf{\underline{Negrito com sublinhado}} \newline
Aqui um exemplo de \textbf{\textit{\underline{Negrito com italico e sublinhado}}} \newline
Aqui um exemplo de \emph{ênfase em um trecho} do texto \newline

\textbf{Observação:} Observe que a primeira linha dos exemplos acima "Automaticamente" ficou adiantada (Inicio de paragrafo) e veja que abaixo não ficou... Para isto, foi utilizado o comando "noindent" no inicio do parágrafo (Neste caso, da linha). \newline


\noindent Aqui um exemplo de \textbf{negrito} \newline
Aqui um exemplo de \textit{Italico} \newline
Aqui um exemplo de \underline{Sublinhado} \newline
Aqui um exemplo de \textbf{\textit{Negrito com italico}} \newline
Aqui um exemplo de \textbf{\underline{Negrito com sublinhado}} \newline
Aqui um exemplo de \textbf{\textit{\underline{Negrito com italico e sublinhado}}} \newline
Aqui um exemplo de \emph{ênfase em um trecho} do texto


\newpage
\section{Como definir espaçamento vertical entre linhas}

Esta é a primeira linha. Agora imagine que voce precisa que a segunda comece com 1.53 cm de distancia abaixo daqui.
\vspace{1.53cm}

Aqui esta a segunda linha, 2.18cm de distancia abaixo. Se quiser mais uma ultima linha, PERFEITAMENTE ao final desta página...
\vspace{\fill} %Sempre distancie a proxima linha abaixo a 1 ENTRE de distancia desta!

Esta linha ficara sempre no final da pagina.

\newpage
\section{Como trabalhar com parágrafos, alinhamento e recuo de texto}

Veja os exemplos abaixo:

\paragraph{Sem definir alinhamento:}
Este texto está justificado automaticamente e vejam a que interessante. Este texto está justificado automaticamente e vejam a que interessante. Este texto está justificado automaticamente e vejam a que interessante. 

\paragraph{Alinhando à esquerda}
\begin{flushleft}
Este texto está alinhado à esquerda e vejam a que interessante. Este texto está alinhado à esquerda e vejam a que interessante. Este texto está alinhado à esquerda e vejam a que interessante. 
\end{flushleft}

\paragraph{Alinhando à direita}
\begin{flushright}
Este texto está alinhado à direita e vejam a que interessante. Este texto está alinhado à direita e vejam a que interessante. Este texto está alinhado à direita e vejam a que interessante. 
\end{flushright}

\paragraph{Alinhando ao centro}
\begin{center}
Este texto está alinhado ao centro e vejam a que interessante. Este texto está alinhado ao centro e vejam a que interessante. Este texto está alinhado ao centro e vejam a que interessante. 
\end{center}

\paragraph{Definindo recuo personalizado:}
Imagine que voce precise definir um recuo maior para seu texto... veja exemplo abaixo:


\newpage
\section{Como fazer grandes blocos de comentários}

Para isto, veja o codigo referente à esta seção. Há um bloco de texto aqui embaixo que está no código do documento, mas que não nao aparecerá... \newline

\iffalse % -- Inicio do bloco comentado --

Tudo que esta aqui nao aparece...
$$\int_{-\infty}^\infty\mathrm{d}x\,x^{-2}$$ 
Esta será a ultima linha oculta

\fi % % -- Fim do bloco comentado --

Acima deste texto, há um bloco de conteúdo que esta oculto pelos comandos iffalse e fi !!!

\newpage
\section{Como fazer citações recuadas}

Esta seção não aborda as citações das referências bibliográficas. Aborda aqueles trechos de texto RECUADOS devidos as citações. Existem dois comandos basicos:

\begin{enumerate}
\item Comando quote
\item Comando quotation
\end{enumerate}

\paragraph{}
Exemplo com quote:

\begin{quote}
"Aqui foi usado o comando quote Aqui foi usado o comando quote Aqui foi usado o comando quote Aqui foi usado o comando quote Aqui foi usado o comando quote"
\end{quote} 

\paragraph{}
Exemplo com quotation:

\begin{quotation}
"Aqui foi usado o comando quotation Aqui foi usado o comando quotation Aqui foi usado o comando quotation Aqui foi usado o comando quotation Aqui foi usado o comando quotation Aqui foi usado o comando quotation Aqui foi usado o comando quotation Aqui foi usado o comando quotation "
\end{quotation} 

\newpage
é aço não! freqüência

kurose-artigo \cite{kurose-artigo}
kurose-livro \cite{kurose-livro}
kurose-misc \cite{kurose-site}
kurose-procedings \cite{kurose-inprocedings}



\bibliographystyle{plain}
\bibliography{ref}

\end{document}
